\documentclass[12pt]{article}
\usepackage{graphicx}
\title{Autocorrelation in Weather}
\author{Shiyun Liu}}
\date{10.2018}
\begin{document}
  \maketitle
  \begin{abstract}
    Here in this task, we want to answer the question:
    Are temperatures of one year significantly correlated with the next year (successive years), across years in a given location?
  \end{abstract}
  \section{Introduction}
    The annual mean temperatures in Key West, Florida for the 20th century (100 years), were given in this task.
    And we can find out whether there is a significant correlation in weather in Key West area during 20th century, by using computational methods and tools to calculate the correlation.
  \section{Methods}
    Correlation coefficient between successive years is calculated.
    And the correlation coefficients of 10000 sets of randomly permuted year sequences (based on the same database) are calculated.
    The fraction of the correlation coefficients(random) that are greater than that the ordered correlation calculated, would be the approximate p-value.
  \section{Result}
  \begin{figure}
	    \centering
	    \includegraphics[scale=.3]{../Result/TAutoCorrP.pdf}
	    \caption{Temperature changes during 100 years}
  \end{figure}
    The fraction I got is 6e-04. 
    (The value differs every time we run the script as the ramdomly permuted year sequences are randomly generated.
     But the order of magnitude is stable, which is e-04.)
  \section{Discussion}
    Since the fraction value (P-value) is much smaller than 0.05, we can be at least 95 percent confident to say that our correlation coefficent is reflecting the true relationship instead of being discovered by chance.
    So we can draw the conclusion that the temperature of one year are significantly correlated with their successive years across the 20th century in Key West, Florida.
  \bibliographystyle{plain}
  \bibliography{}
\end{document}
